\chapter{The Hybrid Flip-Flop Qubit}\label{CH:single}

The flip-flop qubit is a hybrid of a spin and a charge qubit~\cite{tosi_2017}.
The physical system is a phosphorus donor implanted in an isotopically enriched silicon substrate close to a Si/SiO$_2$ interface.
A surface metal gate induces an interface quantum dot (iQD), and is also used to tune the applied electric field in the growth($z$)-direction.
The bare flip-flop states for the electron and nuclear spins on the donor, $\ket{\uparrow\Downarrow}$ and $\ket{\downarrow\Uparrow}$, are isolated from the polarized states and are highly coherent.
However, as a pure spin qubit, albeit encoded in a two-spin state, it suffers from the same issues with respect to long-distance communication.
The solution provided by the flip-flop qubit design is to adjust the surface gate potentials so that the donor electron can tunnel to the iQD with a tunable magnitude.
The electron locating at the iQD or the donor would form the basis for a charge qubit.
With hyperfine interaction only present on the donor site, and the electron $g$-factor different between the donor and iQD sites, the spin and charge qubits are coupled by these magnetic inhomogeneities, which allow electrical control of the spin states.
The dressed flip-flop qubit is then defined as the lowest two eigenstates of this coupled spin-charge system.

In this chapter, we explore the single isolated flip-flop qubit.
We first examine the bare charge and spin parts of the system.
Then we hybridize the system by coupling the spin to the charge and explore the evolution of the the system in parameter-space.

\section{Charge Qubit}\label{SEC:single_charge}

% Hamiltonian
% Wavefunctions
% Critical Field
\section{Spin Qubit}\label{SEC:single_spin}

% Zeeman
% Hyperfine
\section{Spin-Charge Hybridization}\label{SEC:single_hybrid}

\subsection{Coupling Charge to Spin}
\subsection{Eigenenergies and Bases}
\subsection{Sweet Spot}
\section{Charge Noise}\label{SEC:single_charge_noise}

% 1/f spectrum
\section{Time Evolution}\label{SEC:single_solution}

\subsection{Evolution in a Noiseless Environment}
\subsection{Evolution in a Noisy Environment}