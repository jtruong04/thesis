\chapter{Methods}\label{CH:methods}

In this chapter, we describe a couple techniques that are used and referred to throughout the rest of the paper.
The first technique is the Schrieffer-Wollf transformation, also known as Lowdin partitioning or quasi-degenerate perturbation theory~\cite{winkler_2003}, in which we reduce a larger Hilbert space to a smaller more interesting space.
This is typically used to eliminate any undesired degrees of freedom.
The second technique is used to solve for the time dependency of the density matrix of a system.
This is done by solving the von Neumann equation and using a cumulant expansion to take the ensemble average of the system.

\section{Schrieffer-Wolff Transformation}\label{SEC:sw_trans}

\subsection{Matrix Element Approach}
\subsection{Operator Approach}
\subsection{An Example: Donor Spin Coupling}
\subsection{Matrix Element Approach}

This approach is best suited for Hamiltonians of finite Hilbert space where we can split the space into exactly two sets, the desired set $A$ and undesired set $B$.

We rewrite equation~\ref{EQ:sw_trans_expanded} as follows,
\begin{equation}
    \tilde{H}_d + \tilde{H}_n = \sum_j \frac{1}{j!}\commute{H_0+H_1+H_2}{S}^{(j)} \,,
\end{equation}
where $\tilde{H}_d$ ($\tilde{H}_n$) represents the block (non-block) diagonal part of $\tilde{H}$.
Furthermore, by design, $\tilde{H}_n$ must be zero to our desired order of perturbation theory.
Since $S$ must be similar in structure to $H_2$, we can then write the following equations
\begin{align}
    \tilde{H}_d & = \sum_j \frac{1}{(2j)!}\commute{H_0+H_1}{S}^{(2j)} + \frac{1}{(2j+1)!}\commute{H_2}{S}^{(2j+1)}     \label{EQ:sw_trans_hd} \\
    \tilde{H}_n & = \sum_j \frac{1}{(2j+1)!}\commute{H_0+H_1}{S}^{(2j+1)} + \frac{1}{(2j)!}\commute{H_2}{S}^{(2j)} = 0 \label{EQ:sw_trans_hn}
\end{align}
We can then solve for $S$ by using equation~\ref{EQ:sw_trans_hn} and then finally for $\tilde{H}$ but subsituting $S$ into equation~\ref{EQ:sw_trans_hd}. Here, we list $\tilde{H} = \tilde{H}^{(0)} + \tilde{H}^{(1)}+ \tilde{H}^{(2)}+...$ up to second order.
\begin{subequations}
    \begin{align}
        H^{(0)}_{mm'} & = H_{0,mm'}                                                                        \\
        H^{(1)}_{mm'} & = H'_{mm'}                                                                         \\
        H^{(2)}_{mm'} & = \frac{1}{2}\sum_l H'_{ml}H'_{lm'}\left[\frac1{E_m-E_l}+\frac1{E_{m'}-E_l}\right]
    \end{align}
\end{subequations}
where $H' = H_1 + H_2$, $m$ and $m'$ correspond to states in the desired subset $A$, and $l$ corresponds to states in subset $B$.

\subsection{Operator Approach}

The above method may not work for infinite Hilbert spaces or even just larger Hilbert spaces where we limited by computational power.
The following approach is well suited to large or infinite systems that can be separated into smaller finite sized spaces of equal dimensions where the coupling between blocks is well known.

An example of this is a finite system coupled to a microwave cavity. Let us begin with a Hamiltonian in the form
\begin{equation}
    H = H_{0,sys} + \hbar\omega_c a^\dagger a + Va^\dagger + V^\dagger a \,,
\end{equation}
where $H_{0,sys}$ is the unperturbed Hamiltonian acting only on the finite system, $\omega_c$ is the cavity frequency, $V$ acts only on the system and describe the coupling of the system to the cavity, and  $a$ ($a^\dagger$) is the annihilation (creation) operator for the cavity photons. The last two terms in the above equation serve the role of $H_2$ in equation~\ref{EQ:sw_trans_init_ham}.
For the sake of simplicity, we'll also assume $H_1 = 0$.

Since $S$ must have a similar form to $H_2$, we can write an ansatz for $S$,
\begin{equation}
    S = \hat{S}a^\dagger - \hat{S}^\dagger a \,.
\end{equation}
The minus sign is because $S$ must be anti-Hermitian.
From equation~\ref{EQ:sw_trans_expanded}, we can determine $\tilde{H}$ order by order,
\begin{align}
    \tilde{H}^{(0)} & = H_0                                                                                                                                                             \\
    \tilde{H}^{(1)} & = Va^\dagger + V^\dagger a + \commute{H_0}{S} = 0                                                                                                                 \\
    \tilde{H}^{(2)} & = \commute{Va^\dagger + V^\dagger a}{S} + \frac{1}{2}\commute{\commute{H_0}{S}}{S} = \frac{1}{2}\commute{Va^\dagger + V^\dagger a}{S} \label{EQ:sw_trans_h_order}
\end{align}
We then use the fact that $\tilde{H}^{(1)}=0$ to solve for $S$.
\begin{align*}
    \commute{H_{0,sys}+\hbar \omega_c a^\dagger a}{\hat{S}a^\dagger - \hat{S}^\dagger a} & = -Va^\dagger - V^\dagger a
\end{align*}
Solving for $\hat{S}$ yields
\begin{align}
    \hat{S}_{mn}         & = -\frac{V_{mn}}{E_m-E_n+\hbar\omega_c}         \\
    \hat{S}^\dagger_{mn} & = \frac{V^\dagger_{mn}}{E_m-E_n-\hbar \omega_c}
\end{align}
where the $E_m$ and $E_n$ are the eigenenergies of the system. We can now substitute back into \ref{EQ:sw_trans_h_order} to get the transformed Hamiltonian. Neglecting $a^2$ and $(a^\dagger)^2$ terms, the final second order effective Hamiltonian is
\begin{align}
    \tilde{H}^{(2)} & = \frac{1}{2}\commute{Va^\dagger + V^\dagger a}{\hat{S}a^\dagger - \hat{S}^\dagger a}                                                     \nonumber \\
                    & = \frac{1}{2}\left(\commute{ V^\dagger a}{\hat{S}a^\dagger }-\commute{Va^\dagger }{\hat{S}^\dagger a}\right)\nonumber                     \nonumber \\
                    & = \frac{1}{2}\left(V^\dagger\hat{S}+\hat{S}^\dagger V+\left(\commute{ V^\dagger }{\hat{S}}-\commute{V}{\hat{S}^\dagger}\right)a^\dagger a\right)
\end{align}

\subsection{An Example: Donor Spin Coupling}

In order to illustrate the usefulness of the Schrieffer-Wolff transformation, we apply it to the Kane quantum computer~\cite{kane_1998} in order to obtain the effective coupling between phosphorus donor spins.
In this model, each donor spin is coupled to their respective donor electrons via the contact hyperfine interaction.
The two electrons are then coupled to each other via the exchange interaction.
This in turn yields a higher order effectively coupling between the nuclear spins.
The Schrieffer-Wolff transformation then can be used to calculate the form of this interaction.

\begin{figure}[htbp]
    \centering
    % TODO: Add figure. Energy spectrum for Kane QC
    \includegraphics[width=0.75\columnwidth]{placeholder-image}
    \caption[Kane quantum computer energy spectrum.]{The energy spectrum for the Kane quantum computer. Note the well separated energy manifolds. The lowest energy manifold is the one of interest.}
    \label{FIG:kane_qc_spectrum}
\end{figure}

The model can be represented by the Hamiltonian (with $\hbar=1$)
\begin{equation}
    H = -\sum_{i=1,2} \left(\frac{1}{2}\Omega_i \tau_{z}^{(i)} + \frac{1}{2}\omega_i \sigma_z^{(i)} - A_i \vec{\tau}^{(i)}\cdot \vec{\sigma}^{(i)}\right) + J\vec{\sigma}^{(1)}\cdot\vec{\sigma}^{(2)} \,,
\end{equation}
where $\vec{\tau}^{(i)}$ ($\vec{\sigma}^{(i)}$) are the Pauli matrices for the nuclear (electron) spins, $\Omega_i$ ($\omega_{i}$) are the Zeeman energy splittings, $A_i$ are the hyperfine coupling strength for each nuclear-electron spin pair, and $J$ is the electron exchange interaction.
This is a sixteen level system.
However, since the electron Zeeman energy is much greater than the nuclear Zeeman energy, this system is separated into three well-separated energy manifolds corresponding to the three electron spin configurations with different energy (the electron states $\ket{\uparrow\downarrow}$ and $\ket{\downarrow\uparrow}$ have the same energy).
We can then readily see that the hyperfine interactions couple the parallel electron spin manifolds ($\ket{\uparrow\uparrow}$, $\ket{\downarrow\downarrow}$) with the antiparallel manifold ($\ket{\uparrow\downarrow}$, $\ket{\downarrow\uparrow}$) while the exhange interaction couples within the antiparallel manifold.
The Schrieffer-Wolff transformation can be used to reduce this sixteen dimensional system down to a four dimensional system by restricting our space to the two-electron ground state, $\ket{\downarrow\downarrow}$.

We begin the Schrieffer-Wolff transformation by defining our $A$ and $B$ subsets and splitting the total Hamiltonian into the unperturbed Hamiltonian $H_0$ and perturbation $H'$,
\begin{align}
    H_0 & = -\frac{1}{2}\sum_i \left(\Omega_i \tau_{z}^{(i)} + \omega_i \sigma_z^{(i)} \right)                                   \\
    H'  & = \sum_i \left( A_i \vec{\tau}^{(i)}\cdot \vec{\sigma}^{(i)}\right) + J\vec{\sigma}^{(1)}\cdot\vec{\sigma}^{(2)}   \,,
\end{align}
with $A$ being the set of four states in the two-electron ground state and $B$ containing the remaining twelve states.
Following the procedure and equations outlined above up to the third order, we arrive at an effective Hamiltonian,
\begin{multline}
    H_\textrm{eff}= -\frac{1}{2}\sum_i \left( \Omega_i -2 A_i - \frac{4A_i^2(2J+\omega_i-\Omega_i)}{(\omega_i-\Omega_i)^2} \right)\tau_z^{(i)} \\ + 2JA_1A_2\left( \frac{1}{(\omega_1-\Omega_1)(\omega_2-\Omega_1)} + \frac{1}{(\omega_1-\Omega_2)(\omega_2-\Omega_2)}\right)(\tau_x^{(1)}\tau_x^{(2)}+\tau_y^{(1)}\tau_y^{(2)}) \,,
\end{multline}
where the constant term has been omitted.
The Schrieffer-Wolff transformation then allows us to easily see that using the electron exchange and hyperfine interactions as the intermediary between the nuclear spins yields two main effects.
Firstly, the hyperfine interaction introduces a phase shift which is further modified by the exchange coupling.
Secondly, and more importantly, the effective coupling between the nuclear spins is in the form of an $XY$ interaction with strength proportional to $JA_1A_2$ which can be used to build iSWAP gates.

\section{Solution to the von Neumann Equation}\label{SEC:von_neumann}
\subsection{Decay Profile: $1/f$ Noise}

To reiterate, in order to determine the effect of the noise spectrum on the system, we need to compute the decay profile function,
\begin{equation*}
    J_n(t,\omega_1,\omega_2) = \int_{t_0}^tdt_1e^{i\omega_1t_1}\int_{t_0}^{t_1}dt_2 e^{i\omega_2t_2} S_n(t_1-t_2) \,.
\end{equation*}
We first split the above equation into two parts, a symmetric and an antisymmetric part,
\begin{align}
    J_{nS}(t,\omega_1,\omega_2) & = \frac{1}{2}\int_{t_0}^tdt_1\int_{t_0}^{t_1}dt_2 (e^{i\omega_1t_1}e^{i\omega_2t_2}+e^{i\omega_2t_1}e^{i\omega_1t_2}) S_n(t_1-t_2) \\
    J_{nA}(t,\omega_1,\omega_2) & = \frac{1}{2}\int_{t_0}^tdt_1\int_{t_0}^{t_1}dt_2 (e^{i\omega_1t_1}e^{i\omega_2t_2}-e^{i\omega_2t_1}e^{i\omega_1t_2}) S_n(t_1-t_2)
\end{align}
such that $J_n(t,\omega_1,\omega_2) = J_{nS}(t,\omega_1,\omega_2) + J_{nA}(t,\omega_1,\omega_2)$.
We can now evaluate these to be
\begin{align}
    J_{nS}(t,\omega_1,\omega_2) & = TODO \\
    J_{nA}(t,\omega_1,\omega_2) & = TODO
\end{align}
where
\begin{subequations}\label{EQ:f_integrals_generic}
    \begin{align}
        f_0(t)              & = TODO \\
        f_t(t)              & = TODO \\
        f_{\sin}(t,\omega)  & = TODO \\
        f_{\cos}(t,\omega)  & = TODO \\
        f_{t\sin}(t,\omega) & = TODO \\
        f_{t\cos}(t,\omega) & = TODO
    \end{align}
\end{subequations}
Determining the decay profile then boils down to computing these six integrals for whatever the noise spectrum may be.

The power spectrum for $1/f$ noise is given by
\begin{equation}
    \tilde{S}(\omega) =
    \begin{cases}
        S_0/|\omega| & \text{if } \omega_l \le |\omega| \le \omega_h \\
        0            & \text{otherwise}
    \end{cases}
\end{equation}
where $\omega_l$ and $\omega_h$ are the low and high-frequency cutoffs. The temporal correlation function is simply the Fourier transform of this,
\begin{equation*}
    S(t) = \int_{-\infty}^\infty \tilde{S}(\omega)e^{i\omega t}d\omega = 2S_0\left(\Ci(\omega_ht)-\Ci(\omega_lt)\right) \,,
\end{equation*}
where $\Ci(x)$ is the cosine integral and defined by
\begin{equation}
    \Ci(x) = -\int_x^\infty dx \frac{\cos x}{x} \,.
\end{equation}
Furthermore, we choose the normalization constant $S_0$ so that $S(t=0) = 1$.
This results in the time correlation function of
\begin{equation}
    S(t) = \frac{\Ci(\omega_ht)-\Ci(\omega_lt)}{\ln(\omega_h/\omega_l)} \,.
\end{equation}
Substituting into equations~\ref{EQ:f_integrals_generic}, we get for $1/f$ noise
\begin{subequations}
    \begin{align}
        f_0(t)              & = TODO \\
        f_t(t)              & = TODO \\
        f_{\sin}(t,\omega)  & = TODO \\
        f_{\cos}(t,\omega)  & = TODO \\
        f_{t\sin}(t,\omega) & = TODO \\
        f_{t\cos}(t,\omega) & = TODO
    \end{align}
\end{subequations}
% TODO analysis of above equations.