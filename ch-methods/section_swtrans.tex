\section{Schrieffer-Wolff Transformation}\label{SEC:sw_trans}
The Schrieffer-Wolff transformation is a technique for reducing large time-independent Hamiltonians into smaller more manageable ones.
It is particularly useful when the system is divided into well separated energy manifolds and we are only interested in one of them.
This transformation is not unlike standard perturbation theory where the final Hamiltonian is diagonalized.
However, the primary difference is that the Shrieffer-Wolff transformation results in a Hamiltonian that is \emph{block} diagonal.
One benefit of this technique is that it does not require non-degeneracy between states within a block.

We assume that the Hilbert space can be subdivided into two or more weakly interacting sets ($A$, $B$, etc), where we are only interested in one of the sets. The primary condition for this subdivision is that a state $\ket{m}$ in set $A$ may not be degenerate with another state $\ket{n}$ in a different state $B$.

We start by expressing the total Hamiltonian as the sum of an unperturbed Hamiltonian $H_0$, and perturbation $V = H_1 + H_2$,
\begin{equation}
    H = H_0 + H_1 + H_2 \,.
    \label{EQ:sw_trans_init_ham}
\end{equation}
The unperturbed Hamiltonian has known eigenvalues $E_n$ and eigenstates $\ket{n}$.
$H_1$ is solely responsible for \emph{intra}-block coupling while $H_2$ is responsible for \emph{inter}-block coupling.

\begin{figure}[htbp]
    \centering
    % TODO
    \includegraphics[width=0.75\columnwidth]{placeholder-image}
    \caption[Shrieffer-Wolff transformation. Matrix diagrams.]{Diagrams showing the general forms of $H_0$, $H_1$, and $H_2$. $H_0$ is purely diagonal, $H_1$ is block diagonal, and $H_2$ is off-block diagonal.}
    \label{<label>}
\end{figure}

The main conceit of the Schrieffer-Wolff transformation is that we can apply a unitary transformation $S$ on $H$,
\begin{equation}\label{EQ:sw_trans}
    \tilde{H} =e^{-S}He^S \,,
\end{equation}
such that the resulting Hamiltonian $\tilde{H}$ decouples the states between the various sets up to the desired order, i.e. $\bra{m}\tilde{H}\ket{n} = 0$ for all $m$ and $n$ in different subsets.

By expanding out the unitary transformation in a Taylor series,
\begin{equation}
    e^{S} = 1 + S + \frac{1}{2}S^2 + ...\,
\end{equation}
and multiplying out equation~\ref{EQ:sw_trans}, we obtain for the transformed Hamiltonian
\begin{equation}\label{EQ:sw_trans_expanded}
    \tilde{H} = H + \commute{H}{S} + \frac{1}{2}\commute{\commute{H}{S}}{S} + ... = \sum_j \frac{1}{j!}\commute{H}{S}^{(j)} \,,
\end{equation}
where the commutators $\commute{A}{B}^{(j)}$ are defined by $[...[[A,\underbrace{B],B]...B}_{j\textrm{ times}}]$.

From this point, we show two potential approaches depending on the form of $H$. The first allows us to compute $\tilde{H}$ element by element. The second allows us to compute $\tilde{H}$ by multiplying operators.